%%%%%%%%%%%%%%%%%%%%%%%%%%%%%%%%%%%%%%%%%
% Structured General Purpose Assignment
% LaTeX Template
%
% This template has been downloaded from:
% http://www.latextemplates.com
%
% Original author:
% Ted Pavlic (http://www.tedpavlic.com)
%
% Note:
% The \lipsum[#] commands throughout this template generate dummy text
% to fill the template out. These commands should all be removed when 
% writing assignment content.
%
%%%%%%%%%%%%%%%%%%%%%%%%%%%%%%%%%%%%%%%%%

%----------------------------------------------------------------------------------------
%	PACKAGES AND OTHER DOCUMENT CONFIGURATIONS
%----------------------------------------------------------------------------------------

\documentclass{article}

\usepackage[brazilian]{babel}
\usepackage[utf8]{inputenc}
\usepackage{fancyhdr} % Required for custom headers
\usepackage{lastpage} % Required to determine the last page for the footer
\usepackage{extramarks} % Required for headers and footers
\usepackage{graphicx} % Required to insert images
\usepackage{float}
\usepackage{listings}
\usepackage{hyperref}

\graphicspath{ {img/} }

% Margins
\topmargin=-0.45in
\evensidemargin=0in
\oddsidemargin=0in
\textwidth=6.5in
\textheight=9.0in
\headsep=0.25in 

\linespread{1.1} % Line spacing

% Set up the header and footer
\pagestyle{fancy}
\lhead{\hmwkAuthorName} % Top left header
%\chead{\hmwkClass\ (\hmwkClassInstructor\ \hmwkClassTime): \hmwkTitle} % Top center header
\rhead{\hmwkClass: \hmwkTitle} % Top center header
%\rhead{\firstxmark} % Top right header
\lfoot{\lastxmark} % Bottom left footer
\cfoot{} % Bottom center footer
\rfoot{Página\ \thepage\ de\ \pageref{LastPage}} % Bottom right footer
\renewcommand\headrulewidth{0.4pt} % Size of the header rule
\renewcommand\footrulewidth{0.4pt} % Size of the footer rule

\setlength\parindent{0pt} % Removes all indentation from paragraphs

%----------------------------------------------------------------------------------------
%	DOCUMENT STRUCTURE COMMANDS
%	Skip this unless you know what you're doing
%----------------------------------------------------------------------------------------

% Header and footer for when a page split occurs within a problem environment
\newcommand{\enterProblemHeader}[1]{
\nobreak\extramarks{#1}{#1 continuação na próxima página\ldots}\nobreak
\nobreak\extramarks{#1 (continuação)}{#1 continuação na próxima página\ldots}\nobreak
}

% Header and footer for when a page split occurs between problem environments
\newcommand{\exitProblemHeader}[1]{
\nobreak\extramarks{#1 (continuação)}{#1 continuação na próxima página\ldots}\nobreak
\nobreak\extramarks{#1}{}\nobreak
}

\setcounter{secnumdepth}{0} % Removes default section numbers
\newcounter{homeworkProblemCounter} % Creates a counter to keep track of the number of problems

\newcommand{\homeworkProblemName}{}
\newenvironment{homeworkProblem}[1][Questão \arabic{homeworkProblemCounter}]{ % Makes a new environment called homeworkProblem which takes 1 argument (custom name) but the default is "Problem #"
\stepcounter{homeworkProblemCounter} % Increase counter for number of problems
\renewcommand{\homeworkProblemName}{#1} % Assign \homeworkProblemName the name of the problem
\section{\homeworkProblemName} % Make a section in the document with the custom problem count
\enterProblemHeader{\homeworkProblemName} % Header and footer within the environment
}{
\exitProblemHeader{\homeworkProblemName} % Header and footer after the environment
}

\newcommand{\problemAnswer}[1]{ % Defines the problem answer command with the content as the only argument
\noindent\framebox[\columnwidth][c]{\begin{minipage}{0.98\columnwidth}#1\end{minipage}} % Makes the box around the problem answer and puts the content inside
}

\newcommand{\homeworkSectionName}{}
\newenvironment{homeworkSection}[1]{ % New environment for sections within homework problems, takes 1 argument - the name of the section
\renewcommand{\homeworkSectionName}{#1} % Assign \homeworkSectionName to the name of the section from the environment argumen
\subsection{\homeworkSectionName} % Make a subsection with the custom name of the subsection
\enterProblemHeader{\homeworkProblemName\ [\homeworkSectionName]} % Header and footer within the environment
}{
\enterProblemHeader{\homeworkProblemName} % Header and footer after the environment
}
   
%----------------------------------------------------------------------------------------
%	NAME AND CLASS SECTION
%----------------------------------------------------------------------------------------

\newcommand{\hmwkTitle}{Trabalho \#1} % Assignment title
\newcommand{\hmwkDueDate}{24 de Abril de 2014} % Due date
\newcommand{\hmwkClass}{INF01142} % Course/class
\newcommand{\hmwkClassFull}{INF01142: Sistemas Operacionais I N} % Course/class
\newcommand{\hmwkClassTime}{} % Class/lecture time
\newcommand{\hmwkClassInstructor}{} % Teacher/lecturer
\newcommand{\hmwkAuthorName}{Fernando Bombardelli da Silva(218324), William Bombardelli da Silva(218324)} % Your name

%----------------------------------------------------------------------------------------
%	TITLE PAGE
%----------------------------------------------------------------------------------------

\title{
%\vspace{2in}
\Large\textmd{\textbf{\hmwkClassFull}}\\
\normalsize{\textbf{\hmwkTitle}}\\
\normalsize\vspace{0.1in}\small{\hmwkDueDate}\\
%\vspace{0.1in}
\large{\textit{\hmwkClassInstructor\ \hmwkClassTime}}
%\vspace{3in}
}

\author{\textbf{\hmwkAuthorName}}
\date{} % Insert date here if you want it to appear below your name

%----------------------------------------------------------------------------------------

\begin{document}

%\maketitle
{\centering
\Large\textmd{\textbf{\hmwkClassFull}}\\
\normalsize{\textbf{\hmwkTitle}}\\
\normalsize\vspace{0.1in}\small{\hmwkDueDate}\\
%\vspace{0.1in}
\large\textbf{\hmwkAuthorName}
%\large{\textit{\hmwkClassInstructor\ \hmwkClassTime}}\\

}
%----------------------------------------------------------------------------------------
%	TABLE OF CONTENTS
%----------------------------------------------------------------------------------------

%\setcounter{tocdepth}{1} % Uncomment this line if you don't want subsections listed in the ToC

%\newpage
%\tableofcontents
%\newpage

% To have just one problem per page, simply put a \clearpage after each problem

%----------------------------------------------------------------------------------------
%	Questão 2
%----------------------------------------------------------------------------------------

\begin{homeworkProblem}[Questão 2]

\begin{itemize}
\item Plataforma 1:
	\begin{itemize}
	\item Processador:
		\begin{itemize}
		\item Intel Core i3 @2,30 GHz
		\item 2 cores com suporte a HT (4 thread cores)
		\end{itemize}
	\item Sistema Operacional:
		\begin{itemize}
		\item GNU/Linux CentOS 6.5
		\item Kernel Linux 2.6.32 (x86\underline{ }64)
		\end{itemize}
	\item Compilador:
		\begin{itemize}
		\item GCC 4.4.7
		\end{itemize}
	\item Ambiente não virtualizado
	\end{itemize}
\item Plataforma 2:
	\begin{itemize}
	\item Processador:
		\begin{itemize}
		\item Intel Core i3 @2,30 GHz
		\item 2 cores com suporte a HT (4 thread cores)
		\end{itemize}
	\item Sistema Operacional:
		\begin{itemize}
		\item GNU/Linux Ubuntu 12.04
		\item Kernel Linux 3.2.0 (x86\underline{ }64)
		\end{itemize}
	\item Compilador:
		\begin{itemize}
		\item GCC 4.6.3
		\end{itemize}
	\item Ambiente não virtualizado
	\end{itemize}
\end{itemize}
\end{homeworkProblem}

%----------------------------------------------------------------------------------------
%	Questão 3
%----------------------------------------------------------------------------------------

\begin{homeworkProblem}[Questão 3]

\begin{tabular}{l | l}
	\hline
	\textbf{Função} & \textbf{Correção} \\ \hline
	mcreate & Correto\\
	myield & Correto\\
	mjoin & Correto\\
	mmutex\_init & Correto\\
	mlock & Correto\\
	munlock & Correto\\
	\hline
\end{tabular}
\end{homeworkProblem}

%----------------------------------------------------------------------------------------
%	Questão 4
%----------------------------------------------------------------------------------------

\begin{homeworkProblem}[Questão 4]

\begin{homeworkSection}{ListElem}

	Estrutura utilizada pelo módulo de lista encadeada para armazenar o dado da lista e controlar o encadeamento.
	\begin{lstlisting}[language=C, frame=single]
typedef struct SListElem ListElem;

struct SListElem{
 void *e;
 ListElem *prev;
 ListElem *next;
};
	\end{lstlisting}
	
\end{homeworkSection}

\begin{homeworkSection}{List}

	Estrutura utilizada como descritor da lista encadeada, contém o apontador de início e fim da lista.
	\begin{lstlisting}[language=C, frame=single]
typedef struct SListDesc{
 ListElem *begin;
 ListElem *end;
} List;
// Operacoes:
List* newList();
void freeList(List* listDescriber);
void listCheckRep(List* listDescriber);
void listAdd(List* listDescriber, void* e, int (*comparator) (void*, void*));
void listAppend(List* listDescriber, void* e);
void* listGet(List* listDescriber, void* e, int (*comparator) (void*, void*));
void listRemove(List* listDescriber, void* e, int(*comparator)(void*,void*));
	\end{lstlisting}
	
\end{homeworkSection}

\begin{homeworkSection}{OrderedQueue}

	O tipo \emph{OrdereQueue} é uma estrutura de dados criada para controlar a fila de \emph{threads} aptas a rodar.\\
	A operação \emph{Enqueue} insere um elemento na fila, porém sua posição depende do parâmetro \emph{comparator} informado, que vai determinar por comparação com os outros nós qual a posição do elemento entrante. Esta operação permite o escalonamento das \emph{threads} pela política Shortest Process Next.\\
	A operação \emph{Dequeue} remove o primeiro elemento da lista.\\
	De fato o descritor é idêntico ao descritor da lista encadeada.
	\begin{lstlisting}[language=C, frame=single, breaklines=true]
typedef List OrderedQueue;
// Operacoes:
OrderedQueue* newOrderedQueue();
void freeOrderedQueue(OrderedQueue* queue);
boolean orderedQueueEmpty(OrderedQueue* queue);
void orderedQueueEnqueue(OrderedQueue* queue, void* e, int (*comparator) (void*, void*));
void* orderedQueueDequeue(OrderedQueue* queue);
	\end{lstlisting}
	
\end{homeworkSection}

\begin{homeworkSection}{mmutex\_t}

	Estrutura da variável de controle mutex.\\
	O campo \emph{flag} é uma enumeração que pode assumir o valor \emph{Locked} ou \emph{Free} indicando o estado da seção crítica que a variável controla.\\
	O campo \emph{ownerThread} contém o ID da \emph{thread} que detém o \emph{lock} da variável. Esse campo é utilizado para evitar que uma \emph{thread} diferente da que tem o \emph{lock} tente liberar a seção crítica.\\
	O campo \emph{waitingQueue} mantém a fila dos IDs das \emph{threads} esperando pela liberação do \emph{lock}.
	\begin{lstlisting}[language=C, frame=single]
typedef enum {Locked, Free} MutexFlag;

typedef struct Smmutex_t{
    MutexFlag flag;
    int ownerThread;
    OrderedQueue* waitingQueue;
} mmutex_t;
	\end{lstlisting}
	
\end{homeworkSection}

\end{homeworkProblem}

%----------------------------------------------------------------------------------------
%	Questão 5
%----------------------------------------------------------------------------------------

\begin{homeworkProblem}[Questão 5 - myield]

\begin{tabular}{l | l | p{10cm}}
	\hline
	\textbf{Momento} & \textbf{Chamada de Sistema} & \textbf{Descrição} \\ \hline
	T1 & getcontext & Salva o contexto da \emph{thread main}. É executada apenas uma vez durante toda a execução do programa, com o intuito de criar as estruturas necessárias para realizar o escalonamento da \emph{thread main} juntamente com as demais.\\ \hline
	T2 & clock\_gettime & Consulta o \emph{timer} do sistema para calcular o tempo executado pela \emph{thread} corrente.\\ \hline
	T3 & getcontext & Salva o contexto da \emph{thread} corrente na sua estrutura TCB. Quando esta for chamada novamente, sua execução retornará  a esse ponto. Note que neste caso o escalonamento não será executado em função do controle da variável global \emph{yielding}.\\ \hline
	T4 & clock\_gettime & Salva o tempo atual em um campo do TCB da \emph{thread} que será posta em execução pelo escalonador. Esse tempo será usado posteriormente para definir o tempo executado pela \emph{thread}.\\ \hline
	T5 & swapcontext & Troca o contexto atual pelo contexto da \emph{thread} selecionada para executar. A partir desse ponto a próxima \emph{thread} entrará em execução.\\
	\hline
\end{tabular}
\end{homeworkProblem}

%----------------------------------------------------------------------------------------
%	Questão 6
%----------------------------------------------------------------------------------------

\begin{homeworkProblem}[Questão 6 - mjoin]

\begin{tabular}{l | l | p{10cm}}
	\hline
	\textbf{Momento} & \textbf{Chamada de Sistema} & \textbf{Descrição} \\ \hline
	T1 & getcontext & Salva o contexto da \emph{thread main}. É executada apenas uma vez durante toda a execução do programa, com o intuito de criar as estruturas necessárias para realizar o escalonamento da \emph{thread main} juntamente com as demais.\\ \hline
	T2 & clock\_gettime & Consulta o \emph{timer} do sistema para calcular o tempo executado pela \emph{thread} corrente.\\ \hline
	T3 & getcontext & Salva o contexto da \emph{thread} corrente na sua estrutura TCB. Quando esta for chamada novamente (após o fim da \emph{thread} a ser esperada), sua execução retornará  a esse ponto. Note que neste caso o escalonamento não será executado em função do controle da variável global \emph{yielding}.\\ \hline
	T4 & clock\_gettime & Salva o tempo atual em um campo do TCB da \emph{thread} que será posta em execução pelo escalonador. Esse tempo será usado posteriormente para definir o tempo executado pela \emph{thread}.\\ \hline
	T5 & swapcontext & Troca o contexto atual pelo contexto da \emph{thread} selecionada para executar. A partir desse ponto a próxima \emph{thread} entrará em execução.\\
	\hline
\end{tabular}

Note que caso uma \emph{thread} execute \emph{mjoin} para esperar por uma \emph{thread} que ja tenha sido encerrada, \emph{mjoin} retorna código de erro -1, seguindo especificação. Note ainda que este é o mesmo comportamento do caso de o id recebido por parâmetro em \emph{mjoin} não seja válido (não exista \emph{thread} com tal id).
\end{homeworkProblem}

%----------------------------------------------------------------------------------------
%	Questão 7
%----------------------------------------------------------------------------------------

\begin{homeworkProblem}[Questão 7 - mlock]

\begin{tabular}{l | l | p{10cm}}
	\hline
	\textbf{Momento} & \textbf{Chamada de Sistema} & \textbf{Descrição} \\ \hline
	T1 & getcontext & Salva o contexto da \emph{thread main}. É executada apenas uma vez durante toda a execução do programa, com o intuito de criar as estruturas necessárias para realizar o escalonamento da \emph{thread main} juntamente com as demais.\\ \hline
	T2 & getcontext & Salva o contexto da \emph{thread} corrente na sua estrutura TCB. Quando esta for chamada novamente sua execução retornará a esse ponto. Note que neste caso o escalonamento não será executado em função do controle da variável global \emph{yielding}.\\ \hline
	T3 & clock\_gettime & Salva o tempo atual em um campo do TCB da \emph{thread} que será posta em execução pelo escalonador. Esse tempo será usado posteriormente para definir o tempo executado pela \emph{thread}.\\ \hline
	T4 & swapcontext & Troca o contexto atual pelo contexto da \emph{thread} selecionada para executar. A partir desse ponto a próxima \emph{thread} entrará em execução.\\
	\hline
\end{tabular}

Note que os passos dos tempos T2, T3 e T4 executam enquanto a variável \emph{mutex} em questão não estiver livre. Ou seja, se na chamada para \emph{mlock} o \emph{mutex} estiver livre, as chamadas de sistema nos tempos T2, T3 e T4 não serão executadas.
\end{homeworkProblem}

%----------------------------------------------------------------------------------------
%	Questão 8
%----------------------------------------------------------------------------------------

\begin{homeworkProblem}[Questão 8 - munlock]

\begin{tabular}{l | l | p{10cm}}
	\hline
	\textbf{Momento} & \textbf{Chamada de Sistema} & \textbf{Descrição} \\ \hline
	T1 & getcontext & Salva o contexto da \emph{thread main}. É executada apenas uma vez durante toda a execução do programa, com o intuito de criar as estruturas necessárias para realizar o escalonamento da \emph{thread main} juntamente com as demais.\\ \hline
	T2 & clock\_gettime & Consulta o \emph{timer} do sistema para calcular o tempo executado pela \emph{thread} corrente. Após isto, muda o estado de uma \emph{thread} que estava esperando pelo \emph{unlock} para apto.\\
	\hline
\end{tabular}

Note que o passo do tempo T2 executa apenas se houver alguma \emph{thread} na fila dos bloqueados por este lock.
\end{homeworkProblem}

%----------------------------------------------------------------------------------------
%	Questão 9
%----------------------------------------------------------------------------------------

\begin{homeworkProblem}[Questão 9]

Os testes da biblioteca seguiram os seguintes passos:
\begin{enumerate}
\item Testes unitários nos módulos de lista encadeada e de fila da biblioteca.
\item Debug passo a passo para funcionalidades essenciais das funções oferecidas pela biblioteca e das funções do escalonador. Nesse passo foi utilizado o GDB integrado à IDE NetBeans.
\item Testes unitários das funções oferecidas pela bilbioteca.
\item Testes de sistema utilizando a biblioteca \emph{libmmthread.a} finalizada juntamente com programas chamadores que exploram diversas funcionalidades.
\end{enumerate}
\end{homeworkProblem}

%----------------------------------------------------------------------------------------
%	Questão 10
%----------------------------------------------------------------------------------------

\begin{homeworkProblem}[Questão 10]

\begin{tabular}{l | p{6cm} | p{6cm}}
	\hline
	\textbf{Dificuldade} & \textbf{Descrição} & \textbf{Solução Utilizada}\\ \hline
	Análise do problema & A análise foi o ponto que exigiu o maior esforço de raciocínio. Uma boa análise e planejamento implicariam certamente uma boa execução do desenvolvimento. & Para facilitar a análise foi utilizado diagramas de classe UML (para descrever os módulos) e diagramas de sequência (para descrever as chamadas de funções internas à biblioteca). A saber, a ferramenta utilizada foi o Astah Community.\\ \hline
	Chamada de sistema & Uma dificuldade bem específica encontrada foi a utilização das chamadas de sistema que facilitassem as trocas de contexto. Primeiramente utilizou-se \emph{setcontext} para carregar um novo contexto, porém em algumas circunstâncias ocorria uma exceção de ponto flutuante dentro da função \emph{setcontext}. Mais detalhes no \href{http://www.intel.com/content/dam/www/public/us/en/documents/manuals/64-ia-32-architectures-software-developer-instruction-set-reference-manual-325383.pdf}{Manual da Intel} página 369. & Utilizar \emph{swapcontext} para a troca de contexto. \\ \hline
	Depuração & Erros inesperados naturalmente ocorreram, dado que é muito difícil prever todas as situações no momento da análise. & Para depurar o código e resolver tais erros foi utilizado o GDB integrado à IDE Netbeans.\\
	\hline
	Testes & Outra dificuldade encontrada foi desenvolver casos de testes capazes de comprovar a correção e a completude, ou seja, a partir da especifiação ela faz o que é proposto de meneira correta. & Tentou-se projetar casos de testes de maneira a alcançar um maior \emph{coverage} de código, e mostrar que não há contradição entra a especificação e a implementação.
\end{tabular}
\end{homeworkProblem}

\end{document}

